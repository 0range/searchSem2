\documentclass[12pt]{article}
\usepackage[utf8]{inputenc}
\usepackage[russian]{babel} %comment it for english!
\usepackage{amsfonts,longtable,amssymb,amsmath,array}
\usepackage{graphicx}
\usepackage{euscript}
\usepackage{graphicx}
\usepackage{xcolor}
\usepackage{hyperref}
\graphicspath{ {images/} }
\definecolor{linkcolor}{HTML}{799B03} % цвет ссылок
\definecolor{urlcolor}{HTML}{799B03} % цвет гиперссылок 
\hypersetup{pdfstartview=FitH,  linkcolor=linkcolor,urlcolor=urlcolor, colorlinks=true}
\newtheorem{vkTheorem}{Theorem}[section]
\newtheorem{vkLemma}[vkTheorem]{Lemma}
\newenvironment{vkProof}%
   {\par\noindent{\bf Proof.\par\nopagebreak}}%
   {\hfill$\scriptstyle\blacksquare$\par\medskip}
%\textwidth=550pt%%
%\textheight=650pt
%\oddsidemargin=0pt
%\hoffset=0pt
%\voffset=0pt
%\topmargin=0pt
%\headheight=0pt
%\headsep=0pt
\newcommand{\suml}[0]{\sum\limits}
\begin{document}


\section*{Арзуманян Виталий, CS, 2 курс}

\subsection*{Работа на смоделированных данный}
Для параметра параметра $m$ в интервале от 1 до 512 на сгенерированных выборках:

\includegraphics[scale=0.5]{1.png}

Оптимальное значение $m=12$, отмечено белым.

\subsection*{Работа на большом индексе}

\subsubsection*{Предобработка}

Индекс состоит из записей вида: номер слова, номер документа:количество.
Мы проходим по индексу и получаем список массивов, в каждом из которых номер документа повторяется столько раз, сколько в нем встречается заданное слово. Далее список документов упорядочиваетя по возрастанию. На вход для кодирования подаются эти списки. Оптимальный параметр для ускорения подбираем сначала две итерации по степеням двойки в предполагаемом диапазоне, затем с некоторым шагом на отрезке предполагаемого минимума. Минимум рассматриваем для суммарного размера сжатого индекса (не рассчитывая степени сжатия).

Оптимальный параметр - для которого минимальный размер. $m = $

Гистограмма распределения степеней сжатия для этого параметра:

\includegraphics[scale=0.5]{2.png}


\subsection*{Оценка параметра распределения}

Самый длинный постинг лист состоит из $N=$ записей.
Подберем параметры распределения по максимуму правдоподобия:
$$L = \Pi (1-p)^{X_i}p$$
$$l = N\log p + \sum X_i\log(1-p)$$
$$l' = 0 = \frac{N}{p} - \frac{\sum X_i}{1-p}$$
$$N(1-p) = p\sum X_i$$
Получаем оценку максимального правдоподобия:
$$\widehat{p} = \frac{N}{\sum X_i + N}$$

Полученное значение $p=0.944099$.

\subsection*{Оптимальный параметр кода}

Для этого кода оптимальный параметр из перебираемых 512 - максимальный. Степень сжатия 2.78.

\subsection*{Теоретическая оценка}

Теоретическая оценка дает m=1. 

Очевидно, что для некоторых p ($p + p^2 < 1$) будет оценка $m=1$ и так же что при этом кодировании длина кода будет длиной унарной записи числа + 1, что будет длинее начальной записи числа. Следовательно, для таких $p$ теоретическая оценка некорректна - даже простая унарная запись будет более оптимальным кодом.


\begin{flushright}
$\Box$
\end{flushright}

\end{document} 
